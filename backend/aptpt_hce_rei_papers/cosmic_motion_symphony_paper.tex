\documentclass[12pt, twocolumn]{article}
\usepackage{amsmath, amssymb}
\usepackage{natbib}
\usepackage{graphicx}
\usepackage{hyperref}
\usepackage{geometry}
\geometry{a4paper, margin=1in}
\usepackage{caption}
\usepackage{subcaption}
\usepackage{booktabs}
\usepackage{siunitx}
\usepackage{lipsum} % For placeholder text in figure descriptions

\title{Discovery of the Cosmic Motion Symphony: A Universal Pattern of Synchronized Motion Across Astrophysical, Cosmological, and Computational Scales}
\author{James Gleason\thanks{Email: [your email, if applicable]. The concept, ideas, and mathematical framework were conceived by the author. Artificial intelligence (Grok 3, created by xAI) was used to assist with mathematical computations, data analysis, visualization generation, and manuscript drafting.}}
\date{June 17, 2025}

\begin{document}

\maketitle

\begin{abstract}
This study introduces the \textbf{Cosmic Motion Symphony (CMS)}, a novel empirical pattern of synchronized motion networks observed across diverse astrophysical, cosmological, and computational systems, including cosmic rays, stars, galaxies, X-ray sources, cosmic microwave background (CMB) fluctuations, gravitational wave events, molecular clouds, and simulated particles. Conceived by the author, CMS manifests as streams, conical structures, non-local resonances, and rhythmic cycles, derived without reliance on established physical theories. The Spatial-Temporal Resonance (STR) metric, designed by the author, quantifies motion or variability alignment across spatial and temporal scales, with CMS scaling STR by coherence prevalence. Nine public datasets—Pierre Auger Observatory cosmic rays, Gaia DR3 stellar motions, TESS stellar variability, Planck 2018 CMB, NASA/IPAC NED galaxies, HEASARC X-ray sources, LIGO gravitational wave events, ALMA molecular cloud velocities, and a synthetic particle simulation—validate CMS through extensive predictive testing (70–85\% accuracy), bootstrap resampling (95\% CI), Monte Carlo simulations (p < 0.01), sensitivity analyses, temporal stability tests, and cross-dataset pattern matching (average PCC = 0.79). Artificial intelligence (Grok 3, xAI) facilitated computations, data processing, and drafting, enabling the author, a non-traditional researcher with limited resources, to formalize this discovery. CMS suggests a fundamental cosmic order, potentially revealing a new organizational principle independent of known physical laws.
\end{abstract}

% ... (rest of the Cosmic Motion Symphony paper as provided by the user) ...

\end{document} 